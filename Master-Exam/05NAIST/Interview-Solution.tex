%!TEX program = xelatex
% 完整编译: xelatex -> biber/bibtex -> xelatex -> xelatex
\documentclass[lang=cn,11pt,a4paper]{elegantpaper}

\title{奈良先端科学技術大学院大学面接問答集\\(中文、日本語、English)}
\author{MA Shanpeng 馬 善鵬}
\institute{\href{https://inusturbo.github.io/}{個人ホームページ}}

\version{0.1}
\date{\zhtoday}


% 本文档命令
\usepackage{array}
\newcommand{\ccr}[1]{\makecell{{\color{#1}\rule{1cm}{1cm}}}}
\setcounter{secnumdepth}{3}
\setcounter{tocdepth}{2}
\begin{document}

\maketitle

\begin{abstract}

\keywords{奈良先端大,NAIST,面接,問答集}

\end{abstract}

 \tableofcontents

\section{经常出现的提问|全般的によくある質問|General FAQs}
\subsection{为什么选择NAIST|NAISTを選ぶ理由|Why NAIST}
\subsubsection{中文}
因为NAIST是一所专注于科学研究的学校。特别是NAIST的学生可能有着各种各样的背景,比如艺术类的背景,我看到加藤教授研究室也有艺术学学位毕业的同学,我想多学科背景的交叉可能会碰撞出更多可能。
\subsubsection{日本語}
NAISTは科学的研究に重点を置いた学校ですから。 特にNAISTの学生は美術系など様々なバックグラウンドを持っているでしょうし、加藤先生の研究室にも美術系を卒業した学生がいるようですから、多分野のバックグラウンドがぶつかり合って、より多くの可能性が生まれるのではないでしょうか。
\subsubsection{English}
Because NAIST is a school that focuses on scientific research. In particular, NAIST students may have a variety of backgrounds, such as art backgrounds, and I saw that Professor Kato's Lab also has students who graduated with degrees in art, so I think the intersection of backgrounds may collide to create more possibilities.

\subsection{为什么选择在日本学习|日本留学の魅力|Why Japan}
\subsubsection{中文}
1.日本的环境让人安心,能够让我更好的投入到科研中。

2.特别是日本有着更加深厚的产学研结合,研究的东西会更加可能的投入应用中。

3.我拥有日语N2,而且刚刚参加了N1的考试。

4.想要学习和XR相关的主题。
\subsubsection{日本語}
1.日本の環境は安心感があり、より研究に没頭できる。

2.特に日本は産学研の融合が深く、研究されたものが応用される可能性が高い。

3.日本語はN2ですが、N1の試験を受けたばかりです。

4.XRに関連するトピックを勉強したい。
\subsubsection{English}
1. The environment in Japan is more reassuring and allows me to be more involved in research.

2. Especially, Japan has a deeper combination of industry, academia and research, and what I research will be more likely to be put into application.

3. I have Japanese N2, and I just took the N1 exam.

4. I want to study topics related to XR.

\subsection{为什么选择这个研究室|この研究室を選ぶ理由|Why this lab}
\subsubsection{中文}
1.因为这个研究室是研发了ARTtoolKit的加藤教授的研究室。在刚刚接触3维相关技术的时候,我就了解到了加藤教授的这项工作,让我一直对能在加藤教授研究室做研究充满了向往。

2.因为这个研究室研究的内容与我之前研究的内容相关,我可以更快的熟悉和了解要研究的内容。

3.我一直以来对3维的图像和图形非常感兴趣,对如何能在各种场景中应用XR相关技术一直有期待。

4.随着通信技术和元宇宙等概念的发展,我认为XR和3维相关的技术是非常有未来前景的。
\subsubsection{日本語}
1.この研究室は、ARTtoolKitを開発した加藤教授の研究室であるため。 私が初めて3D関連技術に触れた時に加藤先生の研究を知り、加藤先生の研究室で研究したいと思うようになりました。

2.この研究室での研究は、今まで自分が研究してきたことと関連しているので、これから自分が研究することに早く慣れ、理解することができる。

3.もともと3次元の映像やグラフィックに興味があり、XR関連技術が様々なシーンで応用できるのではと期待していた。

4.通信技術やメタバースなどの概念の発展により、XRや3次元関連の技術は将来的に非常に有望だと思います。
\subsubsection{English}
1. Because this research lab is the research lab of Professor Kato, who developed ARTtoolKit. I learned about Prof. Kato's work when I was first introduced to 3-D related technology, and I always wanted to do research in Prof. Kato's lab.

2. Because the content of this research lab is related to my previous research, I can get familiar with and understand the content to be researched more quickly.

3. I have always been interested in 3-dimensional images and graphics, and I have been looking forward to how I can apply XR-related technologies to various scenes.

4. With the development of communication technology and concepts such as metaverse, I think XR and 3-dimensional related technologies are very promising for the future.

\subsection{关于编程经历|プログラミング体験について|Programming experience}
\subsubsection{中文}
从2017年大学入学开始学习C语言,随后在课程上学习了C++和Java SQL Scala HTML CSS Javascript,自学了Python和Go语言。现在总计写过的代码应该是超过30000行。\\
写过的东西:\\
为某展会开发过智能大棚控制前端。\\
为了能及时收到来自学校网页的通知,写了一个小爬虫,每当学校发送新通知后会自动给我邮箱发送信息。\\
为班级统计信息方便,设计了简陋的在线问卷系统。\\
科研中,写过SLAM的前端和后端,学习了OpenCV PCL OpenMVS OpenMVG和Eigen3 等库\\
\subsubsection{日本語}
2017年大学入学時にC言語を学び始め、その後C++とJava SQL Scala HTML CSS Javascriptをコースで学び、PythonとGo言語を独学で学ぶ。 今書いているコードの総行数は3万行を超えるだろう。 \\
Things written: \\
展示会用スマートシェッドコントロールフロントエンドを開発。 \\
学校のウェブページからタイムリーな通知を受け取るための小さなクローラーを書き、学校が新しい通知を送るたびに私のメールに自動的にメッセージを送るようにしました。 \\
授業統計情報の利便性を高めるため、初歩的なオンラインアンケートシステムを設計。 \\
科学研究において、SLAMのフロントエンドとバックエンドを作成し、OpenCV PCL OpenMVS OpenMVGやEigen3 などのライブラリを習得しました。
\subsubsection{English}
Started learning C from university entrance in 2017, followed by C++ and Java SQL Scala HTML CSS Javascript on the course, self-taught Python and Go language. Total lines of code written now should be over 30,000. \\
What I have written: \\
Developed a smart shed control front-end for an exhibition. \\
Wrote a small crawler in order to receive timely notifications from the school's webpage, which will automatically send messages to my email whenever the school sends a new notification. \\
Designed a rudimentary online questionnaire system for the convenience of class statistical information. \\
In scientific research, wrote the front-end and back-end of SLAM, learned libraries such as OpenCV PCL OpenMVS OpenMVG and Eigen3 \\

\subsection{擅长的编程语言|得意なプログラミング言語|Expertise in programming languages}
\subsubsection{中文}
C++,因为在计算机视觉领域,许多库是使用C++开发的,比如PCL或者OpenCV,但是现在他们也逐渐有了Python版本,因此我也在学习在Python中应用。
\subsubsection{日本語}
C++です。コンピュータビジョンの分野では、PCLやOpenCVなど、多くのライブラリがC++で開発されていますが、現在はPythonでも徐々に使えるようになってきているので、Pythonでの応用も勉強中です。
\subsubsection{English}
C++, because in the field of computer vision many libraries are developed in C++, such as PCL or OpenCV, but now they are gradually available in Python as well, so I am learning to apply them in Python as well.

\subsection{你是电子商务毕业的,为什么想要学习信息科学|電子商取引学科を卒業されていますが、なぜ情報科学を学ぼうと思ったのでしょうか?|You are an e-commerce graduate, why do you want to study information science}
\subsubsection{中文}
我一直对计算机科学有着很浓厚的兴趣,我虽然是电子商务专业的,但是我完整的修完了计算机科学的所有科目。并且在本科时就有志于参加计算机科学相关的项目。
\subsubsection{日本語}
もともとコンピュータサイエンスに興味があり、電子商取引専攻ですが、コンピュータサイエンスの科目はすべて満遍なく履修しています。 そして、学部生時代にコンピュータサイエンス関連のプロジェクトに参加したことがあります。
\subsubsection{English}
I have always had a strong interest in computer science, and although I am an e-commerce major, I have completed all of the computer science subjects in their entirety. And I have been interested in participating in computer science related projects since I was an undergraduate.

\subsection{毕业后想要做什么工作|修了後のキャリアをどう考えているのか|How do you see your career after completion of the program?}
\subsubsection{中文}
如果有机会的话,我想要进一步进行博士的学习和研究。最后我想要进入公司,因为我想要使实验室的成果走进人们的日常生活中。
\subsubsection{日本語}
もし機会があれば、さらに博士課程に進み、研究を進めたいと考えています。 研究室の成果を人々の暮らしに役立てたいので、ゆくゆくは企業に入りたいと思っています。
\subsubsection{English}
I would like to further my PhD studies and research if I have the opportunity to do so. Eventually I want to enter a company because I want to bring the results of my lab into people's daily lives.

\subsection{数学做出了多少?|数学はどれぐらい出来た?|How well did you do in math?}
\subsubsection{中文}
\subsubsection{日本語}
\subsubsection{English}

\section{关于小论文的提问|小論文に関する質問|Questions about the Paper}
\subsection{用1分钟描述你过去的研究|あなたのこれまでの研究内容を1分間で説明してください。|Describe your past research in 1 minute}
\subsubsection{中文}
\subsubsection{日本語}
\subsubsection{English}

\subsection{用1分钟描述你未来想要进行的研究|今後実施したい研究内容を1分間で説明してください。|Describe in 1 minute the research you want to conduct in the future}
\subsubsection{中文}
\subsubsection{日本語}
\subsubsection{English}

\subsection{用1分钟描述你的小论文|小論文を1分間で説明してください|Describe your mini-essay in 1 minute}
\subsubsection{中文}
\subsubsection{日本語}
\subsubsection{English}

\subsection{想要研究的目的意义是什么?|テーマの魅力について語って?|What is the significance of the purpose of the study?}
\subsubsection{中文}
在过去我使用AR的时候,总是感到几个问题,AR中的物体与真实物体因为视差视距的关系总有不真实的感觉。物体无法被放置在正确的位置上。第二是,AR技术还是以接收为主,如果有机会作为普通用户,也可以创造AR世界,或者增加AR物体,那就好了。因此需要用低成本的方式去建立三维模型。
\subsubsection{日本語}
過去にARを使用した際、視差距離の関係でARのオブジェクトがリアルに感じられないという問題がいつも何度かありました。 オブジェクトを正しい位置に配置することができなかった。 もうひとつは、AR技術はまだレセプションベースなので、通常のユーザーとしてARの世界を作ったり、ARのオブジェクトを追加したりする機会があればいいなと思います。 そのため、低コストで3Dモデルを作成する方法が求められています。
\subsubsection{English}
When I used AR in the past, I always felt several problems, the objects in AR always feel unrealistic with the real objects because of the parallax distance. And the objects cannot be placed in the right position. The second is that AR technology is still mainly received. It would be good if there is an opportunity to create AR worlds or increase AR objects as an ordinary user. Therefore, a low-cost way to create 3D models is needed.

\subsection{什么启发了你这项研究?|この研究に興味を持ったきっかけはなに?|What sparked your interest in this research??}
\subsubsection{中文}
\subsubsection{日本語}
\subsubsection{English}

\subsection{对这项研究的热情的源泉是什么?|研究に対する熱意の源泉はなにか|What is the source of your enthusiasm for research?}
\subsubsection{中文}
\subsubsection{日本語}
\subsubsection{English}

\subsection{这项研究不一定在NAIST做对吗?|それはNAISTじゃなくてもできるよね?|You don't have to be at NAIST to do that, right?}
\subsubsection{中文}
\subsubsection{日本語}
\subsubsection{English}

\subsection{已经有相关研究了不是吗?|こんなのもうあるじゃん|There already are something like this.}
\subsubsection{中文}
\subsubsection{日本語}
\subsubsection{English}

\subsection{这项研究的价值是什么?|このテーマに価値はあるの?|Is there any value in this theme?}
\subsubsection{中文}
\subsubsection{日本語}
\subsubsection{English}

\subsection{简述你的参考文献|参考文献を簡単に説明する|Briefly describe your references}
\subsubsection{中文}
\subsubsection{日本語}
\subsubsection{English}

\section{关于学过内容的提问|修学内容に関する質問|Questions about the content of the course of study}


\section{关于毕业论文的提问|卒論に関する質問|Questions about your thesis}


\section{关于信息学基础的提问|情報基礎に関する質問|Questions about Information Fundamentals}


\section{向教授的提问|教授への質問|Questions to the Professor}
\subsection{有什么想要提问的吗|何か質問はありますか?|Do you have any questions?}
\subsubsection{中文}
我看到加藤教授研究室之前也有三维获取技术的相关研究,但是为什么后来停止了?
\subsubsection{日本語}
加藤先生の研究室では、以前から3D撮影技術に関連する研究を行っていたようですが、なぜその後やめてしまったのでしょうか?
\subsubsection{English}
I saw that Prof. Kato's laboratory also had research related to 3D acquisition technology before, but why did it stop later?
\section{其他|その他|Others}
\subsection{研究领域有名的会议和期刊有哪些?|研究分野で有名な学会/論文誌は?|What are the famous conferences and journals in the field of research?}
\subsubsection{中文|日本語|English}
CVPR (Computer Vision and Pattern Recognition)

ICCV(International Conference on Computer Vision)

IJCV(International Journal of Computer Vision)

\section{致谢|謝辞|Acknowledgement}
\subsubsection{中文}
感谢  \href{https://elegantlatex.org/}{ElegantLaTeX团队} 提供的 \href{https://github.com/ElegantLaTeX/ElegantPaper}{ElegantPaper模板}。使我排版这个文档变得十分便利。

感谢\href{https://hatodove22.notion.site/16f0eba93f3c4153bd1f770892aaf6b1}{面接対策},为我想到各种问题提供了很好的参考。
\subsubsection{日本語}
\href{https://elegantlatex.org/}{ElegantLaTeXチーム}による\href{https://github.com/ElegantLaTeX/ElegantPaper}{ElegantPaperテンプレート}の提供に感謝します。 このドキュメントをレイアウトするのがとても楽になりました。

\href{https://hatodove22.notion.site/16f0eba93f3c4153bd1f770892aaf6b1}{面接対策}の方には、いろいろな質問を考えたときに、よい参考となるものを提供していただき、ありがとうございました。
\subsubsection{English}
Thanks to the \href{https://elegantlatex.org/}{ElegantLaTeX Team}  for the\href{https://github.com/ElegantLaTeX/ElegantPaper}{ElegantPaper template} . It made it very convenient for me to typeset this document.

Thanks to the \href{https://hatodove22.notion.site/16f0eba93f3c4153bd1f770892aaf6b1}{Interview Solution}, which provided a good reference for me to think of various problems.


\nocite{*}
%\printbibliography[heading=bibintoc, title=\ebibname]

%\appendix
%\appendixpage
%\addappheadtotoc

\end{document}
