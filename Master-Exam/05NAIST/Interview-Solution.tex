%!TEX program = xelatex
% 完整编译: xelatex -> biber/bibtex -> xelatex -> xelatex
\documentclass[lang=cn,11pt,a4paper]{elegantpaper}

\title{奈良先端科学技術大学院大学面接問答集\\(中文、日本語、English)}
\author{MA Shanpeng 馬 善鵬}
\institute{\href{https://inusturbo.github.io/}{個人ホームページ}}

\version{0.1}
\date{\zhtoday}


% 本文档命令
\usepackage{array}
\newcommand{\ccr}[1]{\makecell{{\color{#1}\rule{1cm}{1cm}}}}
\setcounter{secnumdepth}{3}
\setcounter{tocdepth}{2}
\begin{document}

\maketitle

\begin{abstract}

\keywords{奈良先端大,NAIST,面接,問答集}

\end{abstract}

 \tableofcontents

\section{经常出现的提问|全般的によくある質問|General FAQs}
\subsection{为什么选择NAIST|NAISTを選ぶ理由|Why NAIST}
\subsubsection{中文}
因为NAIST是一所专注于科学研究的学校。特别是NAIST的学生可能有着各种各样的背景,比如艺术类的背景,我看到加藤教授研究室也有艺术学学位毕业的同学,我想多学科背景的交叉可能会碰撞出更多可能。
\subsubsection{日本語}
NAISTは科学的研究に重点を置いた学校ですから。 特にNAISTの学生は美術系など様々なバックグラウンドを持っているでしょうし、加藤先生の研究室にも美術系を卒業した学生がいるようですから、多分野のバックグラウンドがぶつかり合って、より多くの可能性が生まれるのではないでしょうか。
\subsubsection{English}
Because NAIST is a school that focuses on scientific research. In particular, NAIST students may have a variety of backgrounds, such as art backgrounds, and I saw that Professor Kato's Lab also has students who graduated with degrees in art, so I think the intersection of backgrounds may collide to create more possibilities.

\subsection{为什么选择在日本学习|日本留学の魅力|Why Japan}
\subsubsection{中文}
1.日本的环境让人安心,能够让我更好的投入到科研中。

2.特别是日本有着更加深厚的产学研结合,研究的东西会更加可能的投入应用中。

3.我拥有日语N2,而且刚刚参加了N1的考试。

4.想要学习和XR相关的主题。
\subsubsection{日本語}
1.日本の環境は安心感があり、より研究に没頭できる。

2.特に日本は産学研の融合が深く、研究されたものが応用される可能性が高い。

3.日本語はN2ですが、N1の試験を受けたばかりです。

4.XRに関連するトピックを勉強したい。
\subsubsection{English}
1. The environment in Japan is more reassuring and allows me to be more involved in research.

2. Especially, Japan has a deeper combination of industry, academia and research, and what I research will be more likely to be put into application.

3. I have Japanese N2, and I just took the N1 exam.

4. I want to study topics related to XR.

\subsection{为什么选择这个研究室|この研究室を選ぶ理由|Why this lab}
\subsubsection{中文}
1.因为这个研究室是研发了ARTtoolKit的加藤教授的研究室。在刚刚接触3维相关技术的时候,我就了解到了加藤教授的这项工作,让我一直对能在加藤教授研究室做研究充满了向往。

2.因为这个研究室研究的内容与我之前研究的内容相关,我可以更快的熟悉和了解要研究的内容。

3.我一直以来对3维的图像和图形非常感兴趣,对如何能在各种场景中应用XR相关技术一直有期待。

4.随着通信技术和元宇宙等概念的发展,我认为XR和3维相关的技术是非常有未来前景的。
\subsubsection{日本語}
1.この研究室は、ARTtoolKitを開発した加藤教授の研究室であるため。 私が初めて3D関連技術に触れた時に加藤先生の研究を知り、加藤先生の研究室で研究したいと思うようになりました。

2.この研究室での研究は、今まで自分が研究してきたことと関連しているので、これから自分が研究することに早く慣れ、理解することができる。

3.もともと3次元の映像やグラフィックに興味があり、XR関連技術が様々なシーンで応用できるのではと期待していた。

4.通信技術やメタバースなどの概念の発展により、XRや3次元関連の技術は将来的に非常に有望だと思います。
\subsubsection{English}
1. Because this research lab is the research lab of Professor Kato, who developed ARTtoolKit. I learned about Prof. Kato's work when I was first introduced to 3-D related technology, and I always wanted to do research in Prof. Kato's lab.

2. Because the content of this research lab is related to my previous research, I can get familiar with and understand the content to be researched more quickly.

3. I have always been interested in 3-dimensional images and graphics, and I have been looking forward to how I can apply XR-related technologies to various scenes.

4. With the development of communication technology and concepts such as metaverse, I think XR and 3-dimensional related technologies are very promising for the future.

\subsection{关于编程经历|プログラミング体験について|Programming experience}
\subsubsection{中文}
从2017年大学入学开始学习C语言,随后在课程上学习了C++和Java SQL Scala HTML CSS Javascript,自学了Python和Go语言。现在总计写过的代码应该是超过30000行。\\
写过的东西:\\
为某展会开发过智能大棚控制前端。\\
为了能及时收到来自学校网页的通知,写了一个小爬虫,每当学校发送新通知后会自动给我邮箱发送信息。\\
为班级统计信息方便,设计了简陋的在线问卷系统。\\
科研中,写过SLAM的前端和后端,学习了OpenCV PCL OpenMVS OpenMVG和Eigen3 等库\\
\subsubsection{日本語}
2017年大学入学時にC言語を学び始め、その後C++とJava SQL Scala HTML CSS Javascriptをコースで学び、PythonとGo言語を独学で学ぶ。 今書いているコードの総行数は3万行を超えるだろう。 \\
Things written: \\
展示会用スマートシェッドコントロールフロントエンドを開発。 \\
学校のウェブページからタイムリーな通知を受け取るための小さなクローラーを書き、学校が新しい通知を送るたびに私のメールに自動的にメッセージを送るようにしました。 \\
授業統計情報の利便性を高めるため、初歩的なオンラインアンケートシステムを設計。 \\
科学研究において、SLAMのフロントエンドとバックエンドを作成し、OpenCV PCL OpenMVS OpenMVGやEigen3 などのライブラリを習得しました。
\subsubsection{English}
Started learning C from university entrance in 2017, followed by C++ and Java SQL Scala HTML CSS Javascript on the course, self-taught Python and Go language. Total lines of code written now should be over 30,000. \\
What I have written: \\
Developed a smart shed control front-end for an exhibition. \\
Wrote a small crawler in order to receive timely notifications from the school's webpage, which will automatically send messages to my email whenever the school sends a new notification. \\
Designed a rudimentary online questionnaire system for the convenience of class statistical information. \\
In scientific research, wrote the front-end and back-end of SLAM, learned libraries such as OpenCV PCL OpenMVS OpenMVG and Eigen3 \\

\subsection{擅长的编程语言|得意なプログラミング言語|Expertise in programming languages}
\subsubsection{中文}
C++,因为在计算机视觉领域,许多库是使用C++开发的,比如PCL或者OpenCV,但是现在他们也逐渐有了Python版本,因此我也在学习在Python中应用。
\subsubsection{日本語}
C++です。コンピュータビジョンの分野では、PCLやOpenCVなど、多くのライブラリがC++で開発されていますが、現在はPythonでも徐々に使えるようになってきているので、Pythonでの応用も勉強中です。
\subsubsection{English}
C++, because in the field of computer vision many libraries are developed in C++, such as PCL or OpenCV, but now they are gradually available in Python as well, so I am learning to apply them in Python as well.

\subsection{你是电子商务毕业的,为什么想要学习信息科学|電子商取引学科を卒業されていますが、なぜ情報科学を学ぼうと思ったのでしょうか?|You are an e-commerce graduate, why do you want to study information science}
\subsubsection{中文}
我一直对计算机科学有着很浓厚的兴趣,我虽然是电子商务专业的,但是我完整的修完了计算机科学的所有科目。并且在本科时就有志于参加计算机科学相关的项目。
\subsubsection{日本語}
もともとコンピュータサイエンスに興味があり、電子商取引専攻ですが、コンピュータサイエンスの科目はすべて満遍なく履修しています。 そして、学部生時代にコンピュータサイエンス関連のプロジェクトに参加したことがあります。
\subsubsection{English}
I have always had a strong interest in computer science, and although I am an e-commerce major, I have completed all of the computer science subjects in their entirety. And I have been interested in participating in computer science related projects since I was an undergraduate.

\subsection{毕业后想要做什么工作|修了後のキャリアをどう考えているのか|How do you see your career after completion of the program?}
\subsubsection{中文}
如果有机会的话,我想要进一步进行博士的学习和研究。最后我想要进入公司,因为我想要使实验室的成果走进人们的日常生活中。
\subsubsection{日本語}
もし機会があれば、さらに博士課程に進み、研究を進めたいと考えています。 研究室の成果を人々の暮らしに役立てたいので、ゆくゆくは企業に入りたいと思っています。
\subsubsection{English}
I would like to further my PhD studies and research if I have the opportunity to do so. Eventually I want to enter a company because I want to bring the results of my lab into people's daily lives.

\subsection{数学做出了多少?|数学はどれぐらい出来た?|How well did you do in math?}
\subsubsection{中文}
大概**\%
\subsubsection{日本語}
約 **\%
\subsubsection{English}
About **\%

\section{关于小论文的提问|小論文に関する質問|Questions about the Paper}
\subsection{用1分钟描述你过去的研究|あなたのこれまでの研究内容を1分間で説明してください。|Describe your past research in 1 minute}
\subsubsection{中文}
我在毕业论文中由SLAM技术启发,利用ORB和BoW设计了一款基于树莓派的可以识别地铁零部件的设备。该设备用来防止地铁巡检人员在巡检时漏检。相比于基于深度学习的方法,我的方法训练成本低,无需标记,并且识别准确。
\subsubsection{日本語}
SLAM技術にヒントを得て、Raspberry PiをベースにORBとBoWを使って地下鉄の部品を識別する装置を卒論で設計しました。 この装置は、検査員が巡回中に検査漏れを起こさないようにするために使用されます。 私の手法は、ディープラーニングを用いた手法と比較して、コストが低く、マーカーも不要で、認識精度も高いです。
\subsubsection{English}
In my thesis, inspired by SLAM technology, I designed a Raspberry Pi-based device that can identify subway parts using ORB and BoW. The device is used to prevent subway inspectors from missing inspections during their inspections. Compared to deep learning-based methods, my method has low training cost, requires no markers, and is accurate in recognition.

\subsection{用1分钟描述你未来想要进行的研究|今後実施したい研究内容を1分間で説明してください。|Describe in 1 minute the research you want to conduct in the future}
\subsubsection{中文}
现在的AR大多只能让用户被动的接收信息,用户若想在AR空间中创建自己的三维模型比较困难。而且AR中的3维模型往往由于光照等影响不够真实。因此我的研究想要使用民用级别的RGB-D设备,使用神经网络来获得高质量的3维模型。并且从各个角度都映射正确的光照信息。
\subsubsection{日本語}
現在のARの多くは、ユーザーが受動的に情報を受け取るだけであり、ユーザーが自らAR空間内に3Dモデルを作成することは困難である。 また、ARにおける3次元モデルは、ライティングなどの効果により、十分にリアルでないことが多い。 そこで、私の研究では、普通のRGB-Dデバイスを使い、ニューラルネットワークを使って高品質な3Dモデルを得ることを目的としています。 そして、あらゆる角度から正しいイルミネーション情報をマッピングすること。
\subsubsection{English}
Most of the current AR can only let users passively receive information, and it is more difficult for users to create their own 3D models in AR space. Moreover, the 3D models in AR are often not realistic enough due to lighting and other effects. Therefore, my research wants to use civilian  RGB-D devices and use neural networks to obtain high quality 3D models. And to map the correct lighting information from every angle.

\subsection{想要研究的目的意义是什么?|テーマの魅力について語って?|What is the significance of the purpose of the study?}
\subsubsection{中文}
在过去我使用AR的时候,总是感到几个问题,AR中的物体与真实物体因为视差视距的关系总有不真实的感觉。物体无法被放置在正确的位置上。第二是,AR技术还是以接收为主,如果有机会作为普通用户,也可以创造AR世界,或者增加AR物体,那就好了。因此需要用低成本的方式去建立三维模型。
\subsubsection{日本語}
過去にARを使用した際、視差距離の関係でARのオブジェクトがリアルに感じられないという問題がいつも何度かありました。 オブジェクトを正しい位置に配置することができなかった。 もうひとつは、AR技術はまだレセプションベースなので、通常のユーザーとしてARの世界を作ったり、ARのオブジェクトを追加したりする機会があればいいなと思います。 そのため、低コストで3Dモデルを作成する方法が求められています。
\subsubsection{English}
When I used AR in the past, I always felt several problems, the objects in AR always feel unrealistic with the real objects because of the parallax distance. And the objects cannot be placed in the right position. The second is that AR technology is still mainly received. It would be good if there is an opportunity to create AR worlds or increase AR objects as an ordinary user. Therefore, a low-cost way to create 3D models is needed.

\subsection{什么启发了你这项研究?|この研究に興味を持ったきっかけはなに?|What sparked your interest in this research??}
\subsubsection{中文}
虽然我没有真正接触过关于AR的研究,但我根据我有限的经验来看,在AR中3D物体往往不够真实,这可能是由于光照、反射、还有视距视差导致的透视关系错误。而从2021年开始越来越火的NeRF中,可以根据视角的变化而变化光照和反射。而且是因为基于三维的物体,所以可以更好的矫正视距和视差问题。因此我就想这项技术能不能应用在AR领域。并且我还在思考能不能基于这项技术,实现无标志物的更加自然的AR交互。
\subsubsection{日本語}
私はARの研究にあまり触れていませんが、私の限られた経験では、ARでは3Dオブジェクトが十分にリアルでないことが多く、それはおそらく照明や反射、そして視距離の視差による誤った遠近関係によるものだと思います。 2021年から熱くなるNeRFでは、視点によって照明や映り込みを変えることができます。 また、3次元の物体をベースにしているため、距離や視差の問題をよりよく補正することができます。 そこで、この技術をARに活用できないかと考えたのです。 そして、この技術をベースに、マーカーを使わないより自然なARインタラクションの可能性も考えています。
\subsubsection{English}
Although I have not really been exposed to research on AR, I see from my limited experience that 3D objects in AR are often not realistic enough, which may be due to the wrong perspective relationship caused by illumination, reflection, and also parallax of the view distance. In NeRF, which has been getting hotter since 2021, the lighting and reflection can be changed according to the change in perspective. And because it is based on a 3D object, it can better correct the problem of parallax and distance. So I was wondering if this technology could be applied to the AR field. And I am also thinking about whether this technology can be used to achieve more natural AR interaction without markers.

\subsection{对这项研究的热情的源泉是什么?|研究に対する熱意の源泉はなにか|What is the source of your enthusiasm for research?}
\subsubsection{中文}
我在本科时做过基于点云的去噪和分割,但是苦于一直没有达到理想的效果。看到新的NeRF的效果的时候,我非常震惊,我认为这是一项颠覆3维渲染范式的创新性研究,未来也有很大的研究和应用价值。
\subsubsection{日本語}
学部時代に点群ベースのノイズ除去やセグメンテーションを行いましたが、思うような結果が出ずに苦労しました。 新NeRFの結果を見たときは衝撃を受けましたが、3次元レンダリングのパラダイムを覆す革新的な研究であり、今後の研究・応用価値も大きいと思います。
\subsubsection{English}
I did point cloud-based denoising and segmentation as an undergraduate, but struggled to achieve the desired results. I was very shocked when I saw the effect of the new NeRF, and I think it is an innovative research that overturns the 3-dimensional rendering paradigm, and has great research and application value in the future.

\subsection{这项研究不一定在NAIST做对吗?|それはNAISTじゃなくてもできるよね?|You don't have to be at NAIST to do that, right?}
\subsubsection{中文}
不,Kato教授是研究AR的专家,只有在NAIST进行这项研究才能将NeRF与AR更好的结合起来。并且NAIST有着很多不同专业领域的优秀前辈,我希望能在与他们交流的过程中碰撞出新的想法。
\subsubsection{日本語}
いや、加藤先生はARの専門家ですから、NeRFとARをよりよく融合させるためには、この研究をNAISTでやるしかないんです。 そして、NAISTにはさまざまな専門分野の素晴らしい人たちがたくさんいるので、彼らと交流しながら新しいアイデアを出していけたらと思います。
\subsubsection{English}
No, Prof. Kato is an expert in AR research, and the only way to better integrate NeRF and AR is to do this research at NAIST. And NAIST has many excellent seniors in different fields of expertise, and I hope to collide with them in the process of exchanging new ideas.

\subsection{已经有相关研究了不是吗?|こんなのもうあるじゃん|There already are something like this.}
\subsubsection{中文}
是的,但是目前的研究仅仅是隐式的表达,我想要寻找一个方法来将神经网络隐式渲染的3维物体显式表达出来,并且保留NeRF的优点。这样才能在XR领域应用。
\subsubsection{日本語}
そうですね、でも今の研究はあくまで暗黙的な表現なので、ニューラルネットワークが暗黙的にレンダリングした3次元オブジェクトを明示的に表現し、NeRFのメリットを残す方法を模索したいですね。 これによって、XRでの応用が可能になる。
\subsubsection{English}
Yes, but the current research is only on implicit representation, and I want to find a way to express the 3-dimensional objects implicitly rendered by the neural network explicitly and retain the benefits of NeRF. This way it can be applied in XR field.

\subsection{这项研究的价值是什么?|このテーマに価値はあるの?|Is there any value in this theme?}
\subsubsection{中文}
这项研究可以使AR中的三维物体变得更加真实,并且这是利用民用设备几张照片即可生成3维模型,因此用户可以在AR或者VR空间创建自己的模型。有利于进一步推进XR技术的可用性。
\subsubsection{日本語}
この研究により、ARの3Dオブジェクトをよりリアルにすることができ、民生機器から数枚の写真を使って3Dモデルを生成することで、ユーザーはARやVR空間で自分だけのモデルを作ることができるのです。 XR技術の使い勝手をさらに向上させることに貢献します。
\subsubsection{English}
This research can make 3D objects in AR more realistic, and this is using civilian equipment a few photos can generate 3D models, so users can create their own models in AR or VR space. It is beneficial to further increase the availability of XR technology.

\subsection{简述你的参考文献|参考文献を簡単に説明する|Briefly describe your references}
\subsubsection{中文}
1.论文提出了著名的SIFT特征描述子,并且利用图像金字塔实现了

2. 由于计算机图像处理能力的提升,视觉SLAM领域逐渐受到重视,因此需要构建适配视觉SLAM的回环检测模块。最简单的方法当然是直接提取特征点和描述子进行几何匹配,然而当时主流的SIFT和SURF在特征提取上耗费的时间就已经不可接受,而描述子匹配更为耗时,因此远远达不到SLAM算法对实时性的要求。因此,当时的主流方案是基于图像外观相似度的匹配。在此情况下,这两篇文献提出了方案,可以主要被概括为以下几点:

使用FAST关键点和BRIEF描述子,该方案是当时进行图像特征提取最快的方案

使用层级词袋模型将局部特征描述子聚合成全局特征描述符

对图像检索过程提出了反向索引和正向索引方案,加快了检索的速度

提出了一套回环检测判定算法 

文献提出了一种高效的基于词袋模型的视觉回环检测算法,并且提供了很多可行的工程化建议。后续的词袋模型开源库DBoW3和FBoW等都基于此进行编写与优化。

3. 第一次提出NeRF,NeRF 所做的任务是 Novel View Synthesis(新视角合成),即在若干已知视角下对场景进行一系列的观测(相机内外参、图像、Pose 等),合成任意新视角下的图像。传统方法中,通常这一任务采用三维重建再渲染的方式实现,NeRF希望不进行显式的三维重建过程,仅根据内外参直接得到新视角渲染的图像。为了实现这一目的,NeRF 使用用神经网络作为一个 3D 场景的隐式表达,代替传统的点云、网格、体素、TSDF 等方式,通过这样的网络可以直接渲染任意角度任意位置的投影图像。

4. NeRF给我们一个思路,就是简单的MLP的权重就能表征场景三维结构,所以有后续工作进行了RGBD表面重建。

粗略位姿输入(用网络优化),网络输入xyz和方向角,预测SDF值和RGB,进行更精细的表面重建。

SDF直接用深度相机输入监督;Color处理与NeRF一致。


\subsubsection{日本語}
\subsubsection{English}

\subsection{NeRF的缺点是什么|NeRFのデメリットは何ですか?|What are the disadvantages of NeRF}
\subsubsection{中文}
1.NeRF泛化性较差,一个网络只能代表一个场景——暂无解决方案

2.要极为精确的相机位姿GT——基于GAN、BA层的

3.训练速度较慢,参数量较大——训练加速
\subsubsection{日本語}
1.NeRFの汎化性が悪い、ネットワークは1つのシーンしか表現できない - 解決策はまだない

2. 極めて正確なカメラポーズGT - GAN、BAレイヤーに基づく

3. 学習速度が遅い、パラメータ数が多い - 学習の高速化
\subsubsection{English}
1. NeRF generalization is poor, a network can only represent a scene - no solution yet

2. To be extremely accurate camera bit pose GT - based on GAN, BA layer

3. slow training speed, large number of parameters - training acceleration

\section{关于学过内容的提问|修学内容に関する質問|Questions about the content of the course of study}


\section{关于毕业论文的提问|卒論に関する質問|Questions about your thesis}
ORB(Oriented FAST and Rotated BRIEF)特征也是由关键点和描述子组成。正如其英文全名一样,这种特征使用的特征点是”Oriented FAST“,描述子是”Rotated BRIEF“。其实这两种关键点与描述子都是在ORB特征出现之前就已经存在了,ORB特征的作者将二者进行了一定程度的改进,并将这两者巧妙地结合在一起,得出一种可以快速提取的特征--ORB特征。ORB特征在速度方面相较于SIFT、SURF已经有明显的提升的同时,保持了特征子具有旋转与尺度不变性。

词袋模型(英语:Bag-of-words model)是个在自然语言处理和信息检索(IR)下被简化的表达模型。此模型下,一段文本(比如一个句子或是一个文档)可以用一个装着这些词的袋子来表示,这种表示方式不考虑文法以及词的顺序。最近词袋模型也被应用在电脑视觉领域。




\section{关于信息学基础的提问|情報基礎に関する質問|Questions about Information Fundamentals}
\subsection{什么是AR|ARとは何か|What is AR}
\subsubsection{中文}
增强现实技术是指借助计算机图形技术、可视化技术等技术将虚拟信息叠加集成在真实世界,使得真实世界和虚拟信息同时存在,从而达到超越现实的感官体验。基础技术包括跟踪定位技术、用户交互技术、虚拟融合技术和系统显示技术。
\subsubsection{日本語}
拡張現実感とは、コンピューターグラフィックス技術やビジュアライゼーション技術などを用いて、現実世界に仮想情報を重ね合わせ、現実世界と仮想情報を同時に存在させ、現実を超えた感覚を実現することである。 その基盤技術として、トラッキング・ポジショニング技術、ユーザーインタラクション技術、バーチャルフュージョン技術、システム表示技術などがあります。
\subsubsection{English}
Augmented reality technology refers to the integration of virtual information overlay in the real world with the help of computer graphics technology, visualization technology and other technologies, making the real world and virtual information exist at the same time, so as to achieve a sensory experience beyond reality. The basic technologies include tracking and positioning technology, user interaction technology, virtual fusion technology and system display technology.

\subsection{你对AR有什么了解|ARについて教えてください|What do you know about AR}
\subsubsection{中文}
我认为AR技术有3个特点:1. 真实世界和虚拟世界融合。2. 可以实时交互。 3. 在3维环境中定位物体。
\subsubsection{日本語}
私は、AR技術の特徴として、1.現実世界と仮想世界の融合 2.リアルタイムでのインタラクション 3.現実世界と仮想世界の融合 の3点を挙げています。3. 3次元環境における物体の位置確認
\subsubsection{English}
I think AR technology has 3 characteristics: 1. real world and virtual world integration. 2. the ability to interact in real time. 3. locating objects in a 3D environment.

\section{向教授的提问|教授への質問|Questions to the Professor}
\subsection{有什么想要提问的吗|何か質問はありますか?|Do you have any questions?}
\subsubsection{中文}
我看到加藤教授研究室之前也有三维获取技术的相关研究,但是为什么后来停止了?
\subsubsection{日本語}
加藤先生の研究室では、以前から3D撮影技術に関連する研究を行っていたようですが、なぜその後やめてしまったのでしょうか?
\subsubsection{English}
I saw that Prof. Kato's laboratory also had research related to 3D acquisition technology before, but why did it stop later?
\section{其他|その他|Others}
\subsection{研究领域有名的会议和期刊有哪些?|研究分野で有名な学会/論文誌は?|What are the famous conferences and journals in the field of research?}
\subsubsection{中文|日本語|English}
CVPR (Computer Vision and Pattern Recognition)

ICCV(International Conference on Computer Vision)

IJCV(International Journal of Computer Vision)

\section{致谢|謝辞|Acknowledgement}
\subsubsection{中文}
感谢  \href{https://elegantlatex.org/}{ElegantLaTeX团队} 提供的 \href{https://github.com/ElegantLaTeX/ElegantPaper}{ElegantPaper模板}。使我排版这个文档变得十分便利。

感谢\href{https://hatodove22.notion.site/16f0eba93f3c4153bd1f770892aaf6b1}{面接対策},为我想到各种问题提供了很好的参考。
\subsubsection{日本語}
\href{https://elegantlatex.org/}{ElegantLaTeXチーム}による\href{https://github.com/ElegantLaTeX/ElegantPaper}{ElegantPaperテンプレート}の提供に感謝します。 このドキュメントをレイアウトするのがとても楽になりました。

\href{https://hatodove22.notion.site/16f0eba93f3c4153bd1f770892aaf6b1}{面接対策}の方には、いろいろな質問を考えたときに、よい参考となるものを提供していただき、ありがとうございました。
\subsubsection{English}
Thanks to the \href{https://elegantlatex.org/}{ElegantLaTeX Team}  for the\href{https://github.com/ElegantLaTeX/ElegantPaper}{ElegantPaper template} . It made it very convenient for me to typeset this document.

Thanks to the \href{https://hatodove22.notion.site/16f0eba93f3c4153bd1f770892aaf6b1}{Interview Solution}, which provided a good reference for me to think of various problems.


\nocite{*}
%\printbibliography[heading=bibintoc, title=\ebibname]

%\appendix
%\appendixpage
%\addappheadtotoc

\end{document}
