%!TEX program = xelatex
% 完整编译: xelatex -> biber/bibtex -> xelatex -> xelatex
\documentclass[lang=cn,11pt,a4paper]{elegantpaper}

\title{奈良先端科学技術大学院大学面接問答集\\(中文、日本語、English)}
\author{MA Shanpeng 馬 善鵬}
\institute{\href{https://inusturbo.github.io/}{個人ホームページ}}

\version{0.1}
\date{\zhtoday}


% 本文档命令
\usepackage{array}
\newcommand{\ccr}[1]{\makecell{{\color{#1}\rule{1cm}{1cm}}}}
\setcounter{secnumdepth}{3}
\setcounter{tocdepth}{2}
\begin{document}

\maketitle

\begin{abstract}

\keywords{奈良先端大,NAIST,面接,問答集}

\end{abstract}

 \tableofcontents
 \newpage
\section{经常出现的提问|全般的によくある質問|General FAQs}
\subsection{为什么选择NAIST|Why NAIST}
\subsubsection{中文}
因为NAIST是一所专注于科学研究的学校。特别是NAIST的学生可能有着各种各样的背景,比如艺术类的背景,我看到加藤教授研究室也有艺术学学位毕业的同学,我想多学科背景的交叉可能会碰撞出更多可能。
\subsubsection{English}
Because NAIST is a school that focuses on scientific research. In particular, NAIST students may have a variety of backgrounds, such as art backgrounds, and I saw that Professor Kato's Lab also has students who graduated with degrees in art, so I think the intersection of multidisciplinary backgrounds may collide to create more possibilities.
\subsection{为什么选择在日本学习|Why Japan}
\subsubsection{中文}
1.比起欧美,日本的环境让人安心,能够让我更好的投入到科研中。

2.特别是日本有着更加深厚的产学研结合,研究的东西会更加可能的投入应用中。

3.我拥有日语N2,而且刚刚参加了N1的考试。

4.想要学习和XR相关的主题。
\subsubsection{English}
1. Compared with Europe and the United States, the environment in Japan is more reassuring and allows me to be more involved in research.

2. Especially, Japan has a deeper combination of industry, academia and research, and what I research will be more likely to be put into application.

3. I have Japanese N2, and I just took the N1 exam.

4. I want to study topics related to XR.

\subsection{为什么选择这个研究室|}
\subsubsection{中文}
1.因为这个研究室是研发了ARTtoolKit的加藤教授的研究室。在刚刚接触3维相关技术的时候,我就了解到了加藤教授的这项工作,让我一直对能在加藤教授研究室做研究充满了向往。

2.因为这个研究室研究的内容与我之前研究的内容相关,我可以更快的熟悉和了解要研究的内容。

3.我一直以来对3维的图像和图形非常感兴趣,对如何能在各种场景中应用XR相关技术一直有期待。

4.随着通信技术和元宇宙等概念的发展,我认为XR和3维相关的技术是非常有未来前景的。
\subsubsection{English}
1. Because this research lab is the research lab of Professor Kato, who developed ARTtoolKit. I learned about Prof. Kato's work when I was first introduced to 3-D related technology, and I always wanted to do research in Prof. Kato's lab.

2. Because the content of this research lab is related to my previous research, I can get familiar with and understand the content to be researched more quickly.

3. I have always been interested in 3-dimensional images and graphics, and I have been looking forward to how I can apply XR-related technologies to various scenes.

4. With the development of communication technology and concepts such as metaverse, I think XR and 3-dimensional related technologies are very promising for the future.

Translated with www.DeepL.com/Translator (free version)

\subsection{关于编程经历|}
\subsubsection{中文}

从2017年大学入学开始学习C语言,随后在课程上学习了C++和Java SQL Scala HTML CSS Javascript,自学了Python和Go语言。\\
现在总计写过的代码应该是超过30000行。\\
写过的东西:\\
为某展会开发过智能大棚控制前端。\\
为了能及时收到来自学校网页的通知,写了一个小爬虫,每当学校发送新通知后会自动给我邮箱发送信息。\\
为班级统计信息方便,设计了简陋的在线问卷系统。\\
科研中,写过SLAM的前端和后端,学习了OpenCV PCL OpenMVS OpenMVG和Eigen3 等库\\
\subsubsection{English}
Started learning C from university entrance in 2017, followed by C++ and Java SQL Scala HTML CSS Javascript on the course, self-taught Python and Go language. \\
Total lines of code written now should be over 30,000. \\
What I have written: \\
Developed a smart shed control front-end for an exhibition. \\
Wrote a small crawler in order to receive timely notifications from the school's webpage, which will automatically send messages to my email whenever the school sends a new notification. \\
Designed a rudimentary online questionnaire system for the convenience of class statistical information. \\
In scientific research, wrote the front-end and back-end of SLAM, learned libraries such as OpenCV PCL OpenMVS OpenMVG and Eigen3 \\

\subsection{擅长的编程语言|}
\subsubsection{中文}
\subsection{你是电子商务毕业的,为什么想要学习信息科学|}
\subsubsection{中文}
我一直对计算机科学有着很浓厚的兴趣,我虽然是电子商务专业的,但是我完整的修完了计算机科学的所有科目。并且在本科时就有志于参加计算机科学相关的项目。
\subsubsection{English}
I have always had a strong interest in computer science, and although I am an e-commerce major, I have completed all of the computer science subjects in their entirety. And I have been interested in participating in computer science related projects since I was an undergraduate.


我第一次听说NAIST和Kato教授,是通过了解ARToolKit。
现在随着技术的发展AR可能未来不再需要专门的设备才能体验。通过手机等人人都有的设备就可以实现。
现在
在过去我使用AR的时候,总是感到几个问题,AR中的物体与真实物体因为视差视距的关系总有不真实的感觉。第二,物体无法被放置在正确的位置上。第二是,AR技术还是以接收为主,如果有机会我作为普通用户,也可以创造AR世界,或者增加AR物体,那就好了
- [ ]  研究のモチベーションはなに?
- [ ]  入力と出力はなに?
- [ ]  提案手法がうまく動作するための前提・仮定はなに?
- [ ]  誰が使うことを想定している?
- [ ]  ○○という用語の定義は?
- [ ]  成功条件は?
- [ ]  具体的なユースケースは?
- [ ]  この研究の需要なに?
- [ ]  提案手法のメリットとデメリット説明できる?
- [ ]  引用している文献の内容は簡単に説明できる?
- [ ]  なぜNAISTを目指すのか?お前が研究したい分野で有名な研究室は他の大学にもあるぞ?
- [ ]  なぜほ研究室なのか?
- [ ]  卒業後どうするの?
- [ ]  プログラム何行くらい書いたことある?
- [ ]  卒業研究はなにやってる?
- [ ]  これまでどんな研究・勉強をしてきたのか
- [ ]  それを踏まえて何をしたいのか
- [ ]  何の意味があるのか
- [ ]  それはNAISTじゃないとできないことなのか

\subsection{毕业后想要做什么工作|修了後のキャリアをどう考えているのか}
\subsection{数学做出了多少?|数学はどれぐらい出来た?}
\subsubsection{中文}

\subsubsection{日本語}
\subsubsection{English}
\section{关于小论文的提问|小論文に関する質問|Questions about the Paper}


- [ ]  テーマの魅力について語って?
- [ ]  これはなにを解決する研究?
- [ ]  この研究に興味を持ったきっかけはなに?
- [ ]  研究に対する熱意の源泉はなにか
- [ ]  それはNAISTじゃなくてもできるよね?
- [ ]  こんなのもうあるじゃん
- [ ]  このテーマに価値はあるの?
- [ ]  何を参考にしましたか?



- [ ]  なにを評価するの?
- [ ]  なんのために評価するの?
- [ ]  なんでこの評価手法なの?
- [ ]  どういう結果が得られたらうれしいの?
- [ ]  なにと比較するの?
- [ ]  これで評価は完全なの?
- [ ]  実際にこんな実験できるの?
- [ ]  具体的に取得できるデータは何?
- [ ]  この実験で何が言えるの?



- [ ]  なぜこの機構を採用したの?
- [ ]  マイコンは使ったことある?
- [ ]  実際の使い方が伝わらない,説明して?



- [ ]  入力と出力はなに?
- [ ]  なぜこの手順で計算するの?
- [ ]  代わりに○○じゃだめなの?
- [ ]  アルゴリズム
- [ ]  言語
- [ ]  センサ
- [ ]  計算量のオーダーは?
- [ ]  言語はなに使うの?
- [ ]  いままでどんなプログラム実装したことある?
- [ ]  何行くらい?
- [ ]  なんの言語書ける?
- [ ]  なんでこのアルゴリズム使うの?〇〇の方が良くない?
- [ ]  これ実現するために今勉強していることはなに?
- [ ]  これ実現するためにこれから学ぶ必要があることはなに?



- [ ]  この研究によって誰が得するのか
- [ ]  社会にどのぐらい影響があるか
- [ ]  具体的なユースケースは?
- [ ]  誰が使うことを想定している?


- [ ]  主に参考にした文献はなんですか?どう参考にしましたか?
- [ ]  どうやってそれらの文献を見つけましたか?
- [ ]  研究計画にもっとも近い論文はなんですか?
- [ ]  その論文と比較してどう違うのですか?
- [ ]  〇〇さんの論文を引用していますが,この論文の概要を説明してください
- [ ]  引用した論文以外に読んだ論文について説明してください
- [ ]  どのようにして参考文献を調査しましたか?



- [ ]  この図で示したかったことは何?
- [ ]  図は自分で作ったの?
- [ ]  図はどうやって作ったの?


- [ ]  その後、どう発展させていきたいですか?
- [ ]  何ヶ月ぐらいで実装できると思いますか?
- [ ]  この研究のために今どんな勉強・活動をしていますか?
- [ ]  (専門用語について)簡単に説明して?
- [ ]  研究分野で有名な学会・論文誌は?
- [ ]  このジャーナルは知ってる?


\section{关于学过内容的提问|修学内容に関する質問|Questions about the content of the course of study}


- [ ]  具体的になにしたの?×N
- [ ]  機械学習アルゴリズムについて教えて?


- [ ]  4力学の公式一つ挙げてその説明してください


- [ ]  古典制御って何
- [ ]  現代制御って何
- [ ]  PIDって何


- [ ]  フーリエ変換ってなに?
- [ ]  二分木
- [ ]  熱心に取り組んだ科目は何?

\section{关于毕业论文的提问|卒論に関する質問|Questions about your thesis}
- [ ]  卒論なにしてるの?
- [ ]  新規性はどこにあるの?
- [ ]  なにが嬉しいの?
- [ ]  具体的な手法について説明して?

\section{关于信息学基础的提问|情報基礎に関する質問|Questions about Information Fundamentals}


- [ ]  オブジェクト指向,関数型プログラミング
- [ ]  オーダー
- [ ]  ハッシュ関数
- [ ]  丸め誤差
- [ ]  二分探索
- [ ]  インタプリタとコンパイラの違い
- [ ]  スコープ
- [ ]  マイコンとパソコンの違い
\section{向教授的提问|教授への質問|Questions to the Professor}
\subsection{有什么想要提问的吗|}
\subsubsection{中文}
我看到加藤教授研究室之前也有三维获取技术的相关研究,但是为什么后来停止了?
\section{其他|その他|Others}
\subsection{研究领域有名的会议和期刊有哪些?|研究分野で有名な学会/論文誌は?|What are the famous conferences and journals in the field of research?}
\subsubsection{中文|日本語|English}
CVPR (Computer Vision and Pattern Recognition)

ICCV(International Conference on Computer Vision)

IJCV(International Journal of Computer Vision)

- [ ]  専願?併願?なんで?
- [ ]  英語できる?
- [ ]  英語は何ができる?リスニング?日常会話?
- [ ]  なにかアピールしたいことはある?
- [ ]  趣味やサークルはどんなことやってた?
- [ ]  落ちた場合どうするか考えている?
- [ ]  逆になにか質問ある?
- [ ]  自己アピール
\section{致谢|謝辞|Acknowledgement}
\subsubsection{中文}
感谢  \href{https://elegantlatex.org/}{ElegantLaTeX团队} 提供的 \href{https://github.com/ElegantLaTeX/ElegantPaper}{ElegantPaper模板}。使我排版这个文档变得十分便利。

感谢\href{https://hatodove22.notion.site/16f0eba93f3c4153bd1f770892aaf6b1}{面接対策},为我想到各种问题提供了很好的参考。
\subsubsection{日本語}
\href{https://elegantlatex.org/}{ElegantLaTeXチーム}による\href{https://github.com/ElegantLaTeX/ElegantPaper}{ElegantPaperテンプレート}の提供に感謝します。 このドキュメントをレイアウトするのがとても楽になりました。

\href{https://hatodove22.notion.site/16f0eba93f3c4153bd1f770892aaf6b1}{面接対策}の方には、いろいろな質問を考えたときに、よい参考となるものを提供していただき、ありがとうございました。
\subsubsection{English}
Thanks to the \href{https://elegantlatex.org/}{ElegantLaTeX Team}  for the\href{https://github.com/ElegantLaTeX/ElegantPaper}{ElegantPaper template} . It made it very convenient for me to typeset this document.

Thanks to the \href{https://hatodove22.notion.site/16f0eba93f3c4153bd1f770892aaf6b1}{Interview Solution}, which provided a good reference for me to think of various problems.


\nocite{*}
%\printbibliography[heading=bibintoc, title=\ebibname]

%\appendix
%\appendixpage
%\addappheadtotoc

\end{document}
