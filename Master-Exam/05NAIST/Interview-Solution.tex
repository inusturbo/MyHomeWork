%!TEX program = xelatex
% 完整编译: xelatex -> biber/bibtex -> xelatex -> xelatex
\documentclass[lang=cn,11pt,a4paper]{elegantpaper}

\title{奈良先端科学技術大学院大学面接問答集\\(中国語、日本語、英語)}
\author{MA Shanpeng 馬 善鵬}
\institute{\href{https://inusturbo.github.io/}{個人ホームページ}}

\version{0.1}
\date{\zhtoday}


% 本文档命令
\usepackage{array}
\newcommand{\ccr}[1]{\makecell{{\color{#1}\rule{1cm}{1cm}}}}
\setcounter{secnumdepth}{3}
\setcounter{tocdepth}{2}
\begin{document}

\maketitle

\begin{abstract}

\keywords{奈良先端大,NAIST,面接,問答集}

\end{abstract}

 \tableofcontents
 \newpage
\section{经常出现的提问|全般的によくある質問|General FAQs}
\subsection{Why NAIST}
\subsubsection{中国語}
我第一次听说NAIST和Kato教授,是通过了解ARToolKit。
现在随着技术的发展AR可能未来不再需要专门的设备才能体验。通过手机等人人都有的设备就可以实现。
现在
在过去我使用AR的时候,总是感到几个问题,AR中的物体与真实物体因为视差视距的关系总有不真实的感觉。第二,物体无法被放置在正确的位置上。第二是,AR技术还是以接收为主,如果有机会我作为普通用户,也可以创造AR世界,或者增加AR物体,那就好了

\subsubsection{日本語}
\subsubsection{英語}
\section{关于小论文的提问|小論文に関する質問|Questions about the Paper}

\section{关于报考理由的提问|志望理由に関する質問|Questions about the reason for your application}

\section{关于学过内容的提问|修学内容に関する質問|Questions about the content of the course of study}

\section{关于毕业论文的提问|卒論に関する質問|Questions about your thesis}

\section{关于信息学基础的提问|情報基礎に関する質問|Questions about Information Fundamentals}

\section{向教授的提问|教授への質問|Questions to the Professor}

\section{其他|その他|Others}

\section{致谢|謝辞|Acknowledgement}
\subsubsection{中国語}
感谢  \href{https://elegantlatex.org/}{ElegantLaTeX团队} 提供的 \href{https://github.com/ElegantLaTeX/ElegantPaper}{ElegantPaper模板}。使我排版这个文档变得十分便利。

感谢\href{https://hatodove22.notion.site/16f0eba93f3c4153bd1f770892aaf6b1}{面接対策},为我想到各种问题提供了很好的参考。
\subsubsection{日本語}
\href{https://elegantlatex.org/}{ElegantLaTeXチーム}による\href{https://github.com/ElegantLaTeX/ElegantPaper}{ElegantPaperテンプレート}の提供に感謝します。 このドキュメントをレイアウトするのがとても楽になりました。

\href{https://hatodove22.notion.site/16f0eba93f3c4153bd1f770892aaf6b1}{面接対策}の方には、いろいろな質問を考えたときに、よい参考となるものを提供していただき、ありがとうございました。
\subsubsection{英語}
Thanks to the \href{https://elegantlatex.org/}{ElegantLaTeX Team}  for the\href{https://github.com/ElegantLaTeX/ElegantPaper}{ElegantPaper template} . It made it very convenient for me to typeset this document.

Thanks to the \href{https://hatodove22.notion.site/16f0eba93f3c4153bd1f770892aaf6b1}{Interview Solution}, which provided a good reference for me to think of various problems.


\nocite{*}
%\printbibliography[heading=bibintoc, title=\ebibname]

%\appendix
%\appendixpage
%\addappheadtotoc

\end{document}
